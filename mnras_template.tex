% mnras_template.tex 
%
% LaTeX template for creating an MNRAS paper
%
% v3.0 released 14 May 2015
% (version numbers match those of mnras.cls)
%
% Copyright (C) Royal Astronomical Society 2015
% Authors:
% Keith T. Smith (Royal Astronomical Society)

% Change log
%
% v3.0 May 2015
%    Renamed to match the new package name
%    Version number matches mnras.cls
%    A few minor tweaks to wording
% v1.0 September 2013
%    Beta testing only - never publicly released
%    First version: a simple (ish) template for creating an MNRAS paper

%%%%%%%%%%%%%%%%%%%%%%%%%%%%%%%%%%%%%%%%%%%%%%%%%%
% Basic setup. Most papers should leave these options alone.
\documentclass[fleqn,usenatbib]{mnras}

\usepackage{graphicx}
\usepackage{float}
\usepackage{amsmath}
\usepackage{amssymb}
\usepackage{cite}
\usepackage{natbib}
\usepackage{xcolor}
\usepackage{newtxtext,newtxmath}
\usepackage[T1]{fontenc}
\usepackage{ae,aecompl}


%%%%%%%%%%%%%%%%%%% TITLE PAGE %%%%%%%%%%%%%%%%%%%

% Title of the paper, and the short title which is used in the headers.
% Keep the title short and informative.
\title[SL Challenge]{Strong Gravitational Finding Lensing Challenge}

% The list of authors, and the short list which is used in the headers.
% If you need two or more lines of authors, add an extra line using \newauthor
\author[R. B. Metcalf {\it et al.}]{
R. Benton Metcalf,$^{1}$\thanks{E-mail: robertbenton.metcalf@unibo.it}
A. N. Other,$$
Third Author$$
and Fourth Author$$
\\
% List of institutions
$^{1}$Department of Physics and Astronomy, Universit\`a di Bologna, viale Berti Pichat 6/2, 40127 Bologna, Italy \\
}

% These dates will be filled out by the publisher
\date{Accepted XXX. Received YYY; in original form ZZZ}

% Enter the current year, for the copyright statements etc.
\pubyear{2015}

% Don't change these lines
\begin{document}
\label{firstpage}
\pagerange{\pageref{firstpage}--\pageref{lastpage}}
\maketitle

% Abstract of the paper
\begin{abstract}

\end{abstract}

% Select between one and six entries from the list of approved keywords.
% Don't make up new ones.
\begin{keywords}
keyword1 -- keyword2 -- keyword3
\end{keywords}

%%%%%%%%%%%%%%%%%%%%%%%%%%%%%%%%%%%%%%%%%%%%%%%%%%

%%%%%%%%%%%%%%%%% BODY OF PAPER %%%%%%%%%%%%%%%%%%

\section{Introduction}

\section{The Challenge}

\section{the simulations}

\section{lens finding methods}

\subsection{GAHEC IRAP}

Arcfinder (Alard, 2006, Cabanac et al. 2007, More et al., 2012) is a fast linear method that computes a pixelwise elongation parameter (ratio of first-order moments in a n-pix window oriented in proper reference frame) for all pixels of mexican-hat-smoothed FITS images. Arcfinder then extract contiguous pixels above a given background and computes the candidate arcs length, width, area, radius of curvature and peak surface brightness. A final thresholding is set to maximize purity over completeness on a few typical arcs of the dataset.
For the current SL challenge, arcfinder was tunned to detect long and narrow arcs, and was optimized on a subset of 1000 simulated images with a grid covering a range of elongation windows and arc areas.  A python wrapper allows users to change parameters in a flexible way and run the arcfinder C code as a linux line command. Arcfinder took a couple of hours to run on the entire dataset with some overheads due to the dataset format. The code is publicly available at https://github.com/rcabanac/arcfinder.

optional fig included:
From top-left to botton right, 1) a simulated arc extracted from SL challenge in which an tunned Arcfinder selects 3 candidates (green circles), 2) the smoothed image on which pixelwise elongation is computed, 3) the resulting elongated pixels after thresholding, 4) the set of pixels selected for the computation of arc candidate properties.
\section{results}

\section{Conclusions \& discussion}

\section*{Acknowledgements}

The Acknowledgements section is not numbered. Here you can thank helpful
colleagues, acknowledge funding agencies, telescopes and facilities used etc.
Try to keep it short.

%%%%%%%%%%%%%%%%%%%%%%%%%%%%%%%%%%%%%%%%%%%%%%%%%%

%%%%%%%%%%%%%%%%%%%% REFERENCES %%%%%%%%%%%%%%%%%%

% The best way to enter references is to use BibTeX:

%\bibliographystyle{mnras}
%\bibliography{example} % if your bibtex file is called example.bib


% Alternatively you could enter them by hand, like this:
% This method is tedious and prone to error if you have lots of references
\begin{thebibliography}{99}
\bibitem[\protect\citeauthoryear{Author}{2012}]{Author2012}
Author A.~N., 2013, Journal of Improbable Astronomy, 1, 1
\bibitem[\protect\citeauthoryear{Others}{2013}]{Others2013}
Others S., 2012, Journal of Interesting Stuff, 17, 198
\end{thebibliography}

%%%%%%%%%%%%%%%%%%%%%%%%%%%%%%%%%%%%%%%%%%%%%%%%%%

%%%%%%%%%%%%%%%%% APPENDICES %%%%%%%%%%%%%%%%%%%%%

\appendix

\section{Some extra material}

If you want to present additional material which would interrupt the flow of the main paper,
it can be placed in an Appendix which appears after the list of references.

%%%%%%%%%%%%%%%%%%%%%%%%%%%%%%%%%%%%%%%%%%%%%%%%%%


% Don't change these lines
\bsp	% typesetting comment
\label{lastpage}
\end{document}

% End of mnras_template.tex